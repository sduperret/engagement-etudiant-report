\documentclass{article}
\usepackage[french]{babel}
\usepackage[a4paper, total={6in, 8in}]{geometry}
\usepackage{hyperref}
\hypersetup{
    linktoc=all, %set to all if you want both sections and subsections linked
}
%\usepackage{cite}
%\usepackage{lipsum}

%\usepackage{authblk} %FOR CENTER AUTHOR IMPORTANT !!! DONT ERASE

\title{{\huge UE Engagement étudiant\\Rapport}}
\author{$ $\\{\LARGE Sacha Duperret}\\ $ $\\ \href{mailto:sduperret@u-bordeaux.fr}{sduperret@u-bordeaux.fr}\\Conseiller du collège Sciences et Technologies,\\Université de Bordeaux,\\Talence,\\France}
\date{7 mai 2022}


%Graphicx
\usepackage{graphicx} %Loading the package
\graphicspath{{images/}} %path to figures


\begin{document}
\maketitle

\vspace{20pt}

\tableofcontents

\vspace{50pt}

\begin{abstract}
L'objectif de ce rapport est d'évaluer la capacité à analyser d’une part les activités/tâches exercées dans le cadre d'un.e projet/expérience associatif.ve, et d’autre part à reconnaître la valeur formatrice de cet engagement et plus particulièrement à identifier les compétences et connaissances développées grâce à cette UE et exploitables dans un CV/une lettre de motivation.
\end{abstract}

\newpage \section*{Introduction}

Étudiant à l'université de Bordeaux je me suis engagé en tant qu'élu au Conseil de collège sciences et technologies. 
Cette instance est l’intermédiaire entre le Conseil d'administration de l'université de Bordeaux et les composantes. 
En son sein sont étudiées les modalités de contrôle de connaissances l’internationalisation des formations, les conditions d'admission aux études, la conception et l'organisation de l'offre de formation, et cætera... 
Dix binômes y constituent la représentation des étudiants. 

Mon activité s'est concentrée plus particulièrement sur la facilitation de l'aspect académique de la vie universitaire des étudiants et apprenants. Les actions tendent à rendre l’offre de formation facilement accessible, avec une maquette simple, des frais d'inscription contenus, notamment pour les doubles diplômes, et d'autres. Ce travail se réalise en collaboration avec les autres élus du conseil, les membres de la direction des composantes et celle du collège. 

Depuis décembre 2021, j’ai pu m’engager dans ces objectifs pour œuvrer dans ce sens jusqu’à la fin de mon mandat, que ce soit dans le périmètre du conseil de collège mais aussi dans les réunions et groupes de travail qui en découlent. 

\newpage
Jusqu'ici dans le cadre de mon engagement, mon implication directe s’évalue à une cinquantaine d'heures. 

\section{Conseil de Collège Sciences et Technologies}

\subsection{Description et analyse}
Ma fonction d'élu implique que tout d'abord ma présence lors des réunions du conseils de collège. Durant ces séances mensuelles d'une durée approximative de 4h, différents temps se succèdent, d'informations et de questions, puis de discussion et ensuite de délibération. 

Les questions relèvent des champs très divers. Le périmètre couvre les admissions, les chantiers en cours, les réformes à venir, la vie universitaire et étudiante et bien d'autres sujets. Les réponses sont parfois apportées par des pairs conseillers, l'administration du collège, des intervenants extérieurs (invités) ou quand elles ne sont pas  possibles, les sujets sont consignées pour une recherche ultérieure approfondie. 

Les informations présentées concernent principalement des éléments de calendrier. Y sont mentionnées entre autres les échéances, comme celle de l'accréditation ou de l'inscription en master, les dates de dépôt de dossier, de promotion interne, mais aussi celles des événements (co-)organisés par l'université. Des aspects financiers sont aussi régulièrement évoqués comme la rémunération des personnels, les primes et assimilés. 

Le Conseil remplit aussi son rôle d’instance décisionnaire et dans ce cadre les débats aboutissent à une délibération. 
Cela permet d'acter la position d'accord ou non pour une proposition que le directeur du collège a énoncée ou détaillée. 
Les modalités de vote peuvent varier, mais je n'ai pour l'instant participé qu’à des votes à main levée. \\
Pour assurer la représentativité du Conseil, un quorum a été mis en place. Il correspond à un nombre minimum de participants présents. 
S'il n'est pas atteint aucun vote n'aura lieu, l’assemblée réunie n’aura pas de pouvoir de décision. 
Pour pallier des absences fortuites, un système de procuration est mis en place.

Les discussions constituent une part importante de la durée des séances. Elles provoquent parfois un débat contradictoire et sont nécessaires pour que chacun puisse affiner sa position en vue d'un vote. Tout est lié, chaque sujet peut avoir un impact sur la vie de l'université et des étudiants. Ces temps de discussion sont des moments opportuns pour enrichir le débat de questions concernant des projets, du point de vue étudiant, pour aborder un nouvel aspect ou proposer la création d’un groupe de travail dans le but d'approfondir la réflexion. C’est un temps particulièrement dense nécessitant une forte concentration et un plein investissement, avec une vigilance accrue. En effet des sujets très pointus sont alors évoqués et il est important et plutôt judicieux de s'être renseigné au préalable afin d’éviter des incompréhensions et de mauvais positionnements. Pour cela nous organisons des réunions préparatoires évoquées plus loin dans le document.

Notre travail engagé au conseil de collège porte ses fruits, car avec les autres élus étudiants l'homogénéité de point de vue a permis d’étayer les projets inscrits dans la démarche de renforcement des dispositifs destinés aux étudiants. Pour la partie des finances, le coût d'inscription aux doubles diplômes a été maîtrisé et reste proche du montant des formations universitaires dispensées uniquement par l'Université de Bordeaux. Sur l'aspect vie étudiante, le conseil de collège est réceptif aux problématiques que nous pouvons soulever, dans la mesure où les mesures discutées pourraient avoir un impact direct sur les étudiants, mais il n'est pas l'instance décisionnaire en la matière.

\subsection{Identification des connaissances et compétences acquises}
La première entrée dans la ``salle des Actes''\footnote{\href{http://colloquebordeaux2017.socfjp.com/wp-content/uploads/2016/11/S.-des-Actes-1-768x576.jpg}{lien externe vers une photographie de la salle des actes}} fut un moment privilégié et à caractère solennel. Cette salle au volume imposant résonne des événements passés, associé au prestige d'une architecture au style ancien elle diffuse une ambiance indubitablement respectueuse qu'induit l'Officiel. Elle se situe au premier étage du bâtiment A33. Quel lieu privilégié pour des séances pour des travaux où l’éloquence prend sa pleine vitalité ! \\ Les premières fois, la prise de parole, dans cet environnement  quelque peu déstabilisant et face à un auditoire rompu à l'exercice, est plus que déstabilisante voire effrayante, mais cependant grisante. L'objectif de porter les arguments et de convaincre pour rallier les opinions, permet, avec la concentration, de ramener les choses à l'instant présent. La solennité du lieu impose l'accroissement des capacités de raisonnement, de vision globale et de synthèse des idées, pour apporter un exposé clair et précis sur les sujets débattus. Un moment m'a particulièrement marqué , il étudiait la demande de création d'un cycle pluridisciplinaire d'études supérieures (CPES) dans un partenariat entre le lycée Montaigne et le collège DSPEG (collège Droit science politique économie et gestion). Certains conseillers ont interpellés le présentateur du projet à propos de la distance imposant des délais de transport entre les différents sites de formation et constituant une perte de temps et d'énergie non négligeable. Du versant des représentants des étudiants, la contribution aux interrogations a porté sur la concorde nécessaire en matière d'ajustement financier entre le lycée et l'université, avec un point de vigilance particulier sur le coût d'inscription à cette formation. \\

J'ai pu mesurer l'effet de mon mandat sur le développement des compétences en matière d'éloquence, de structuration des prises de parole, de visions d'ensemble, et de positionnement. D'autres compétences sont sollicitées et développées comme l'agilité mentale et la capacité de reconnexion rapide lors de changement de sujet totalement distinct abordés sans transition. Un exemple a été le passage d'un vote sur des modalités de contrôle de connaissance, à un point d'information sur l'élection du futur directeur du collège sciences et technologies. \\

Sur le plan des connaissances, j'ai pu me familiariser avec l'utilisation d'acronymes, très nombreux à l'université, allant de l'appellation de service à la dénomination d'un programme européen. Je me suis également enrichi des connaissances de chacun.e.s.

%------------------------------------------------------------------------------------------------------
\newpage

\section{Réunions préparatoires} 

\subsection{Description et analyse}
En amont de chaque séance du conseil, une réunion préparatoire est organisée lorsque l'ordre du jour est connu. Ce temps de travail est un moment d'échanges sincères et constructifs avec le directeur et quelques fois ses collaborateurs, comme la responsable administrative et financière du collège. Cette rencontre d'environ une heure permet d'aborder les dossiers présents à l'ordre du jour et de lever quelques interrogations.

Ce temps est essentiel pour mettre en place l'interface entre les étudiant.e.s que nous représentons et l'institution. Les sujets sont extrêmement variés allant de la structuration du site web de présentation de l'offre de formation, vitrine extérieure dédiée aux familles et futurs étudiants, aux travaux en cours ou à venir sur notre campus universitaire. Ces questions vont au delà du vécu ou du ressenti de chacun des élus et il est indispensable de prendre la mesure des questions posées et de leur conséquences dans le quotidien de chaque étudiant.e, ce que nous permet cette préparation. Dernièrement un point important a été mis à l'ordre du jour et a suscité la vigilance des élus étudiants, il concernait l'annonce de la fermeture du Restaurant Universitaire numéro 1 (RU1). L'appréhension des élus s'est immédiatement exprimée sur  la possible durée de la fermeture. L'inquiétude au delà de la nécessité de rénovation et d'amélioration du service rendu, s'est portée sur le point éminemment important d'accès à un repas chaud à prix étudiant. Il ne s'agit pas de pénaliser de nombreuses personnes déjà en situation précaire, ni de provoquer une perte de chance dans leurs études. Cela serait préoccupant sur un plan somatique, physiologique ou psychique et en contradiction avec le rôle bienveillant de l'université.

\subsection{Identification des connaissances et compétences acquises}
Cette réunion demande une préparation différente. 
En effet, elles nécessitent de mener une ``enquête''  auprès des étudiants pour découvrir les sujets à porter, les questionnements que se posent les nouveaux entrants, ... 
Des compétences d'écoute et de synthèse sont donc fortement développées. 
Certains points de vigilance peuvent être les chaînons d'un problème plus épineux qu'il faut identifier. Pour cela je présume avoir développer une vision davantage macroscopique de la vie étudiante et de l'institution, ce qui me permet d'analyser plus finement les tenants et aboutissants des difficultés évoquées précédemment.\par 
Les connaissances développées couvrent un champ très large. 
En outre, l'organisation de la vie étudiante du côté institution est fortement développée, lorsque nous aidons les étudiants à s'orienter dans l'administration, les différents services et les composantes de formations ($\simeq$UF) par exemple. 
La connaissance d'une partie du calendrier institutionnel est un partie non négligeable. 
Elle est parfois nécessaire lors de discussions à délai très court, nécessitant une approbation d'une instance ou d'un service, la Direction des Affaires Juridiques par exemple.
Aussi une connaissance générale, néanmoins utile lors de ces réunions : le champ de compétence du conseil. 
Cela permet de recourir au bon interlocuteur en fonction du sujet. 
De manière élargie, connaître les attributions de chaque service et identifier les personnes ``ressources'' en leur sein permet de développer une réseau de professionnel, autant pour être au courant des grandes lignes des projets des services que pour faire le lien avec les étudiants, et ce dans les deux sens. Sur ce point, j'estime avoir un niveau intermédiaire.
Autre point, non négligeable et nécessaire pour ``avancer'', j'ai appris à chercher. 
Acronymes, textes de loi, procès-verbaux, organigrammes, la plupart des informations qui me manquent sont sur internet, particulièrement sur le site de l'Université ou $legifrance.gouv.fr$. 
Pour la compétences de recherche je considère avoir un niveau plus élevé. J'ai par exemple appris à manipuler le moteur de recherche de Google avec des booléens pour obtenir des résultats ciblés.

%------------------------------------------------------------------------------------------------------
\newpage

\section{Groupes de travail et cætera}

\subsection{Description et analyse}
Aussi nommés GT, les groupes de travail sont des réunions où sont évoquées les évolutions passées et à venir, à l'université. Dernièrement plusieurs ont eu lieu : GT mise en place du portail unique, GT fonctionnement du conseil ST, GT calendrier  universitaire 2022-2023, GT blocs de connaissances et de compétences (BCC). Ces réunions ne sont pas périodiques mais arrivent lors de la mise en place de chantiers ou pour faire un point \textit{``RETEX''} ($=$ retour d'expérience).

J'ai participé au groupe de travail retour d'expérience sur la mise en place du portail unique. Mon rôle était d'apporter aux membres un regard depuis l'intérieur du portail, avec un discussion sur les ``groupes classes'' qui sont différents pour chaque matière et un ``ressenti'' d'étudiant. De possibles améliorations à apporter pour faciliter l'intégration des nouveaux entrants étudiants, la gestion administrative (inscription pédagogique/administrative) et des salles, la mise en groupe ont également été formulées. Des pistes d'améliorations pour 2023-2024 (outils numériques pour la répartition des étudiants, l'importation de données directement dans apogée, ...) ont menées à la mise en place d'autres groupes de travail avec la DSI (Direction des systèmes d’information) de l'université.

Bien sûr, le mail reste le principal outil de travail du conseiller, et plus encore avec la pandémie. Il permet l'organisation de réunions, la discussion, le partage de documents, ... C'est une tâche de fond durant tout le mandat : pour rester au fait des dernières affaires, questions et soucis de la communauté, se tenir au courant des réformes, des chantiers, des cadrages données par l'université et le MESRI (Ministère de l'Enseignement Supérieur et de la Recherche) et, jusqu'à il y a encore quelques semaines, des directives COVID-19 à l'université.

Je l'ai déjà mentionné auparavant, le point principal à maîtriser pour pouvoir comprendre les discussions consiste à assimiler ou deviner les significations des acronymes.
\subsection{Identification des connaissances et compétences acquises}
Lors de groupes de travail, j'ai pu aiguiser mes arguments et affermir ma compétence de représentant de la communauté étudiante face à des auditeurs bienveillants quand à ma méconnaissance du système.\\

Depuis décembre et plus généralement depuis mon entrée à l'université, mon utilisation des mails a beaucoup augmentée. Le plus flagrant étant ma capacité de rédaction (vitesse de frappe, formulation) qui s'est améliorée. Mon vocabulaire a quand à lui évolué grâce cette immersion prolongée dans le monde de l'administration universitaire publique. Pour exemple, le terme marché publique m'est aujourd'hui beaucoup plus familier.\\

Plus généralement, j'ai aujourd'hui une vision autrement plus ``gestionnaire'' que précédemment. Je considère des projets à une échelle plus macroscopique, tout en voulant tendre vers une analyse multiscalaire. Je veux maintenant me diriger vers une analyse pluridisciplinaire des propositions qui sont portées dans le cadre du conseil, en intégrant des volets où j'ai été nouvellement initié, comme par exemple le volet temps RH (HeTD).

%------------------------------------------------------------------------------------------------------ 
\newpage
\begin{center}
\textbf{Répartition approximative des heures d'implications en tant qu'élu étudiant}
\vspace*{10pt}

\begin{tabular}{|c|c|c|}
\hline 
\textbf{Activités} & \textbf{Périodicité} & \textbf{Temps total}\\ 
\hline 
Conseil de collège ST & Tous le mois & 5*4h = 20h \\ 
\hline 
Réunions préparatoires & Tous les mois & 5*1h (direction collège) + 4*1h (candidats) = 9h\\ 
\hline 
Groupes de travail & Non périodique & 1*3h + 1h = 4h\\ 
\hline 
Mail & Plusieurs fois/semaine & 20 semaines*1h/semaine = 20h\\ 
\hline 
\textbf{Total} &  & \textbf{53h} (approximativement) \\ 
\hline 
\end{tabular} 
\end{center}
\textit{NB : * signifie multiplication ; h signifie heure(s)}
	
\section*{Conclusion}
Dans ce rapport, j'ai essayé de retranscrire et analyser les activités, compétences et connaissances que j'ai développé lors de mon engagement en tant que conseiller du collège sciences et technologies.\\

Ce rapport révèle les activités que j'ai réalisé durant ce mandat, passant du conseil de collège aux réunions préparatoires, groupes de travail et finissant par des occupations plus générales.

C'est au niveau de mon éloquence que l'évolution est la plus visible. Mes connaissances de l'institution et des procédés étant aussi affinés par l'expérience. Cependant je reste conscient du travail faramineux qu'il reste à suivre et d'autres compétences à découvrir, comme je l'évoquais en fin de la $partie 3.2$. Dans l'ensemble, je considère que c'est un engagement qui demande disponibilités, investissement et suivi, rapportant plus que des connaissances, il apporte une vision différente de la vie, met en relations des leviers et souligne des assujettissements à faire des compromis, bien présents de nos jours.

A l'avenir, je souhaite aider à lancer une réflexion sur les écarts stridents entre les différents groupes de travaux dirigés (notation, attendus) et certaines modalités de contrôle de connaissances qui ne semblent plus pertinentes de nos jours. Et pourquoi pas me présenter pour m'engager à nouveau...
\newpage
\section{Annexes}
Pour compléter ce rapport, je décris ici les annexes jointes.

\paragraph{Annexe 1} \href{https://nuxeo.u-bordeaux.fr/nuxeo/nxpath/default/default-domain/workspaces/Colleges/Coll\%C3\%A8ge\%20ST.1623307472368/Espace\%20commun\%20coll\%C3\%A8ge\%20ST/Conseil\%20Coll\%C3\%A8ge/2022@view_documents?tabIds=MAIN_TABS\%3Adocuments\%2C\%3A&conversationId=0NXMAIN1}{Lien} vers Nuxeo, interface pour récupérer les documents du conseil (comptes-rendus, procès-verbaux, ordres du jours, annexes, présentations, et d'autres);

\paragraph{Annexe 2} Transparent de présentation du conseil de collège sciences et technologies aux nouveaux conseillers, le 19 mai 2022.
\end{document}