\documentclass{article}
\usepackage[french]{babel}
\usepackage[a4paper, total={6in, 8in}]{geometry}
\usepackage{hyperref}
\hypersetup{
    linktoc=all, %set to all if you want both sections and subsections linked
}
%\usepackage{cite}
%\usepackage{lipsum}

%\usepackage{authblk} %FOR CENTER AUTHOR IMPORTANT !!! DONT ERASE

\title{{\huge UE Engagement étudiant\\Rapport}}
\author{$ $\\{\LARGE Sacha Duperret}\\ $ $\\ \href{mailto:sduperret@u-bordeaux.fr}{sduperret@u-bordeaux.fr}\\Conseiller du collège Sciences et Technologies,\\Université de Bordeaux,\\Talence,\\France}
\date{7 mai 2022}


%Graphicx
\usepackage{graphicx} %Loading the package
\graphicspath{{images/}} %path to figures

%\usepackage{indentfsirst} %to indent section with "alinea" in french

\begin{document}
\maketitle

\vspace{20pt}

\tableofcontents

\vspace{50pt}

\begin{abstract}
L'objectif de ce rapport est d'évaluer la capacité à analyser d’une part les activités/tâches exercées dans le cadre d'un.e projet/expérience associatif.ve, et d’autre part à reconnaître la valeur formatrice de cet engagement et plus particulièrement à identifier les compétences et connaissances développées grâce à cette UE et exploitables dans un CV/une lettre de motivation.
\end{abstract}

\newpage \section*{Introduction}

Étudiant à l'université de Bordeaux je me suis engagé en tant qu'élu au Conseil de collège sciences et technologies. 
Cette instance est l’intermédiaire entre le Conseil d'administration de l'université de Bordeaux et les composantes. 
En son sein sont étudiées les modalités de contrôle de connaissances l’internationalisation des formations, les conditions d'admission aux études, la conception et l'organisation de l'offre de formation, et cætera... 
Dix binômes y constituent la représentation des étudiants. 

Mon activité se concentrait plus particulièrement sur la facilitation de l'aspect académique de la vie universitaire des étudiants et apprenants. 
Les actions tendent à rendre l’offre de formation facilement accessible, avec une maquette simple, des frais d'inscription contenus, notamment pour les doubles diplômes, et d'autres. 
Ce travail se réalise en collaboration avec les autres élus du conseil, les membres de la direction des composantes et celle du collège.

Depuis décembre 2021, j’ai pu m’engager dans ces objectifs pour œuvrer dans ce sens jusqu’à la fin de mon mandat, que ce soit dans le périmètre du conseil de collège mais aussi dans les réunions et groupes de travail qui en découlent. 

\newpage
Jusqu'ici dans le cadre de mon engagement, mon implication directe s’évalue à une cinquantaine d'heures. 

\section{Conseils de Collège Sciences et Technologies}

\subsection{Description et analyse}
Ma fonction d'élu implique tout d'abord ma présence lors des conseils de collège. 
Durant ces séances mensuelles d'une durée approximative de 4h, différents temps se succèdent, d'informations et de questions, puis de discussion et ensuite de délibération. 

Les questions relèvent des champs très divers. 
Le périmètre couvre les admissions, les chantiers en cours, les réformes à venir, la vie universitaire et étudiante et bien d'autres sujets. 
Les réponses sont parfois apportées par des père conseiller y a que par l'administration du collège, des intervenants extérieurs (invités) ou sont consignées pour une recherche ultérieure approfondie. 

Les informations présentées concernent principalement des éléments de calendrier. Y sont mentionnées entre autres les échéances comme celle de l'accréditation ou de l'inscription en master, les dates de dépôt de dossier promotion interne, mais aussi celles des événements (co-)organisés alors l'université. Des aspects financiers sont aussi régulièrement évoqués comme la rémunération des personnels, les primes et assimilés. 

Le Conseil remplit aussi son rôle d’instance décisionnaire et dans ce cadre les débats aboutissent à une délibération. 
Cela permet d'acter la position d'accord ou non pour une proposition que le directeur du collège a énoncée ou détaillée. 
Les modalités de vote peuvent varier, mais je n'ai pour l'instant participé qu’à des votes à main levée. \\
Pour assurer la représentativité du Conseil, un quorum a été mis en place. Il correspond à un nombre minimum de participants présents. 
S'il n'est pas atteint aucun vote n'aura lieu, l’assemblée réunie n’aura pas de pouvoir de décision. 
Pour pallier des absences fortuites, un système de procuration est mis en place. \textit{Certains projets précis comme …, ne nécessitent pas la mise en place d’un tel processus}.

Les discussions constituent une part importante la durée des séances. Elles provoquent parfois un débat contradictoire et sont nécessaire pour que chacun puisse affiner sa position pour un vote. Tout est lié, chaque sujet peut avoir un impact sur la vie de l'université et des étudiants. Ces temps de discussion sont des moments opportuns pour enrichir le débat de questions concernant des projets, du point de vue étudiant, pour aborder un nouvel aspect ou proposer la création d’un groupe de travail dans le but d'approfondir la réflexion. C’est un temps particulièrement dense nécessitant une forte concentration et un plein investissement, avec une vigilance accrue. En effet des sujets très pointus sont alors évoqués et il est important et plutôt judicieux de s'être renseigné au préalable, afin d’éviter des incompréhensions et de mauvais positionnements. Pour cela nous organisons des réunions préparatoires évoquées plus loin dans le document.

Notre travail engagé au conseil de collège porte ses fruits, car avec les autres élus étudiants l'homogénéité de point de vue a permis d’étayer les projets inscrits dans la démarche de renforcement des dispositifs destinés aux étudiants. Pour la partie des finances, le coût d'inscription aux doubles diplômes a été maîtrisé et reste proche du montant concernant les formations universitaires dispensées uniquement par l'Université de Bordeaux . Sur l'aspect vie étudiante, le conseil de collège est réceptif aux problématiques que nous pouvons soulever, dans la mesure où les mesures discutées pourraient avoir un impact, mais  il n'est pas l'instance décisionnaire en la matière.

\subsection{Identification des connaissances et compétences acquises}
Ma première entrée dans la salle des Actes a été un moment assez solennel. 
C'est une grande salle, vestige d'une architecture ancienne et d'une ambiance indubitablement officielle. 
Elle est située au premier étage du bâtiment A33. 
Une des compétences principalement développée pendant les séances est l'éloquence. 
Au début, ce peut être un peu déstabilisant de s'exprimer devant toutes ces personnes et d'essayer de les rallier à notre opinion. 
De plus, nous accroissons notre capacité de raisonnement et notre vision globale des questions évoquées. 
Par exemple durant la demande de création d'un cycle pluridisciplinaire d'études supérieures (CPES) en partenariat avec le lycée Montaigne et le collège DSPEG, certains conseillés ont interpellé le présentateur à propos des temps de transports entre les différents sites sur lesquels se déroule la formation. 
Côté représentants étudiants, une interrogation sur les ajustements financiers entre le lycée et l'université a été mentionnée, ainsi que les coûts d'inscription à la formation.
Pour résumer les compétences principalement développées sont éloquence, structuration des prises de paroles, prises de paroles (et de positions) et vision d'ensemble. \\ 
Nous ne l'avons pas encore examinée, mais une capacité d'adaptation et une forme d'agilité sont développées, avec le rapide passage de sujets foncièrement différent : par exemple d'un vote sur des modalités de contrôle de connaissances à une information sur la prochaine élection du directeur du collège sciences et technologies. \par
\textbf{POINT CONNAISSANCES ICI + NIVEAU A AUTO-ÉVALUER}

%------------------------------------------------------------------------------------------------------
\newpage

\section{Réunions préparatoires} 

\subsection{Description et analyse}
Pour préparer les séances du conseil, une réunion préparatoire est régulièrement organisée avant la tenue de celui-ci. 
C'est un moment d'échange privilégié avec le directeur, et parfois avec la responsable administrative et financière, du collège. 
Pendant cette préparation d'une heure mensuelle, nous suivons l'ordre du jour de la séance à venir et évoquons nos interrogations à ce sujet. 

Une partie importante du temps de discussion est dédié à faire l'interface entre étudiant.e.s et institution. 
Les sujets sont très variés, de la structuration du site web de présentation de l'offre de formations du collège aux travaux qui adviennent sur notre campus. 
Dernièrement un gros point de vigilance porté par les élus étudiants est la fermeture pour travaux du Restaurant Universitaire n°1 (RU1), qui déclenche une appréhension des étudiants (et de leurs élus) concernant nos possibilités en terme de restauration ces prochaines années.
C'est un point important, car il peut précipiter les étudiants aux capacités pécuniaires les plus faibles vers une impossibilité à se nourrir correctement pendant leurs études. 
Ce qui serait préoccupant sur un plan somatique, physiologique ou psychologique. 
La réussite dans leurs études serait aussi diminuée. \\ 
Durant cet échange, il est possible de porter la voix des étudiants ce qui peut amener le collège à initier une réflexion à court, moyen ou long terme.

\subsection{Identification des connaissances et compétences acquises}
Cette réunion demande une préparation différente. 
En effet, il faut principalement mener une "enquête" auprès des étudiants pour découvrir les points de vigilance à porter, les questionnements que se posent les nouveaux entrants, ... 
Donc une compétence d'écoute fortement est développée. 
Pour mieux voir les tenants des points soulevés par les étudiants, il peut être propice de s'imaginer à leur place et de faire preuve d'empathie. 
Sur ces compétences, je trouve qu'il y a encore un travail abondant à faire. 
Je pense en être à un niveau élémentaire. \par 
Les connaissances développées couvrent un champ très large. 
En outre, l'organisation de la vie étudiante du côté institution est fortement développée, lorsque nous aidons les étudiants à s'orienter dans l'administration, les différents services et les composantes de formations ($\simeq$UF) par exemple. 
La connaissance d'une partie du calendrier institutionnel est un partie non négligeable. 
Elle est parfois nécessaire lors de discussions à délai très contenu, nécessitant une approbation d'une instance ou d'un service, la Direction des Affaires Juridiques par exemple.
Aussi une connaissance générale, néanmoins utile lors de ces réunions : le champ de compétence du conseil. 
Cela permet de recourir au bon interlocuteur en fonction du sujet. 
De manière élargie, connaître les attributions de chaque service et identifier les personnes "ressources" en leur sein permet de développer une réseau de professionnel, autant pour être au courant des grandes lignes des projets des services que pour faire le lien avec les étudiants, et dans les deux sens. Sur ce point, j'estime avoir un niveau intermédiaire.
Autre point, non négligeable et nécessaire pour "avancer", j'ai appris à chercher. 
Acronymes, textes de loi, organigrammes, la plupart des informations qui me manquent sont sur internet, particulièrement le site de l'Université. 
Pour la compétences de recherche je considère avoir un niveau élevé, par exemple j'ai appris à manipuler le moteur de recherche Google avec des booléens pour obtenir des résultats plus ciblés.

%------------------------------------------------------------------------------------------------------
\newpage

\section{Groupes de travail et cætera}

\subsection{Description et analyse}
Aussi nommés GT, les groupes de travail sont des réunions où sont discutées les évolutions à l'université. Dernièrement plusieurs ont eu lieu : GT mise en place du portail unique, GT fonctionnement du conseil ST, GT calendrier  universitaire 2022-2023, GT contrôle de connaissances en licence (groupe disciplinaire). Ces réunions ne sont pas périodiques mais arrivent lors de la mise en place de chantiers ou pour faire un point "RETEX" (retour d'expérience).

J'ai participé au groupe de travail retour d'expérience sur la mise en place du portail unique et le ressenti des usagers (= étudiants) après cette première année. Mon rôle était d'apporter un regard de l'intérieur du portail, avec un discussion sur les "groupes classes" qui changent pour chaque matière. Nous avons également formulé des améliorations qui pourraient être apportées pour faciliter l'intégration des nouveaux entrants étudiants, la gestion administrative (inscription pédagogique/administrative) et des salles. Des pistes d'améliorations pour 2023-2024 (outils numériques principalement) ont menées à la mise en place d'autres réunions avec la DSI de l'université.

Bien sûr, le mail reste le principal outil de travail du conseiller, et plus encore avec la pandémie. Il permet l'organisation de réunions, la discussion, le partage de documents, ... C'est une tâche de fond durant tout le mandat : pour rester au fait des dernières affaires, questions et soucis de la communauté, se tenir au courant des réformes en cours, des chantiers, des directions données par l'université et le MESRI (Ministère de l'Enseignement Supérieur et de la Recherche) et, jusqu'à il y a encore quelques semaines, des directives COVID-19 à l'université.

Depuis décembre et plus généralement depuis mon entrée à l'université, mon utilisation des mails a beaucoup augmentée. Le plus flagrant étant ma capacité de rédaction qui s'est améliorée, surtout dans le cadre des fonctions d'élus. Mon vocabulaire a quand à lui évolué grâce cette immersion prolongée dans le monde de l'administration universitaire publique. Pour donner un exemple, le terme marché publique m'est aujourd'hui beaucoup plus familier. 

\textit{Le plus gros travail qui se fait c'est encore l'apprentissage des abréviations spécifiques ;)}
\subsection{Identification des connaissances et compétences acquises}



Cet ensembles d'activités m'a aussi permis de développer des compétences oratoires. Ainsi qu'une vision plus "gestionnaire" qu'avant.
%------------------------------------------------------------------------------------------------------ 
\newpage
\begin{center}
\textbf{Répartition approximative des heures d'implications en tant qu'élu étudiant}
\vspace*{10pt}

\begin{tabular}{|c|c|c|}
\hline 
\textbf{Activités} & \textbf{Périodicité} & \textbf{Temps total}\\ 
\hline 
Conseil de collège ST & Tous le mois & 5*4h = 20h \\ 
\hline 
Réunions préparatoires & Tous les mois & 5*1h (direction collège) + 4*1h (candidats) = 9h\\ 
\hline 
Groupes de travail & Non périodique & 1*3h + 1h = 4h\\ 
\hline 
Mail & Plusieurs fois/semaine & 20 semaines*1h/semaine = 20h\\ 
\hline 
\textbf{Total} &  & \textbf{53h} (approximativement) \\ 
\hline 
\end{tabular} 
\end{center}
\textit{NB : * signifie multiplication ; h signifie heure(s)}

\section*{Conclusion}
\end{document}