\documentclass{article}
\usepackage[french]{babel}
\usepackage[a4paper, total={6in, 8in}]{geometry}
\usepackage{hyperref}
\hypersetup{
    linktoc=all, %set to all if you want both sections and subsections linked
}
\usepackage{cite}
%\usepackage{lipsum}

%\usepackage{authblk} %FOR CENTER AUTHOR IMPORTANT !!! DONT ERASE

\title{{\huge UE Engagement étudiant\\Rapport}}
\author{$ $\\{\LARGE Sacha Duperret}\\ $ $\\ \href{mailto:sduperret@u-bordeaux.fr}{sduperret@u-bordeaux.fr}\\Conseiller du collège Sciences et Technologies,\\Université de Bordeaux,\\Talence,\\France}
\date{7 mai 2022}


%Graphicx
\usepackage{graphicx} %Loading the package
\graphicspath{{images/}} %path to figures

%\usepackage{indentfsirst} %to indent section with "alinea" in french

\begin{document}
\maketitle

\vspace{20pt}

\tableofcontents

\vspace{50pt}

\begin{abstract}
L'objectif de ce rapport est d'évaluer la capacité à analyser d’une part les activités/tâches exercées dans le cadre d'un.e projet/expérience associatif.ve, et d’autre part à reconnaître la valeur formatrice de cet engagement et plus particulièrement à identifier les compétences et connaissances développées grâce à cette UE et exploitables dans un CV/une lettre de motivation.
\end{abstract}

\newpage \section*{Introduction}

Je me suis engagé en tant qu'étudiant élu au conseil de collège sciences et technologies. C'est une instance intermédiaire entre les composantes et le conseil d'administration de l'université de Bordeaux. En son sein, il est discuté et voté les modalités de contrôle de connaissances, l'internationalisation des formations, les conditions d'admissions aux études, la conception et l'organisation de l'offre de formation, et cætera. Nous sommes 9 binômes de représentants des étudiants.

Plus particulièrement, je me concentrais sur la facilitation de l'aspect académique de la vie universitaire des étudiants et apprenants. Cela passe par une offre de formation facilement accessible, une maquette simple, des frais d'inscription contenus pour les doubles diplômes, et d'autres. Bien sur, ce travail se fait en collaboration avec les autres élus du conseil, la direction des composantes et du collège.

Cet objectif a été poursuivi depuis décembre 2021, et je continuerai à le porter jusqu'à la fin de mon mandat au conseil de collège ainsi que dans les réunions, groupes de travail annexes.

\newpage
Dans le cadre de mon engagement étudiant, j'estime m'être impliqué une cinquantaine d'heure.

\section{Conseil de collège ST} 
Tout d'abord, la fonction d'élu.e.s comprend évidemment la présence lors des conseils, que ce soit pour délibérations, discussions ou encore informations. C'est une réunion a périodicité mensuelle, durant environ 4 heures.

Dans le cadre du conseil de collège sciences et technologies : sur un volet général, les résultats sont plutôt bons. Avec les autres élus étudiants une homogénéité des objectifs a permis d'apporter un soutien fort aux projets qui s'inscrivent dans une démarche d'amélioration pour les étudiants. Sur un volet financier, les coût d'inscription aux doubles diplômes sont resté très contenus et similaire aux frais des formation universitaires plus "classiques". Volet vie étudiante, le conseil de collège n'a pas de réelle compétence à ce sujet. Certaines interrogations étaient évoquées durant les questions diverses mais ne pouvaient pas mener à des actions plus concrètent.

\newpage
\section{Réunions préparatoires} 
Pour bien cerner le cadre des discussions, les projets discutés et les conséquences, nous prenons une heure avant chaque conseil. Cela passe par la lecture de l'ordre du jour, permettant une modification mais aussi la pose de questions diverses. C'est un moment d'échange privilégié avec la direction du collège ST. 

Une partie du travail se fait durant la réunion avec la direction du collège, c'est à ce moment là que sont par exemple évoquées les questions d'installation sauvages de personnes sur le campus, du RU n°1 qui ferme, mais également l'accessibilité des informations sur le site du collège. Durant cet échange, il est possible de porter la voix des étudiants (= usagers) et donc d'initier une réflexion à court, moyen ou long terme.

\newpage
\section{Groupes de travail et cætera}
Aussi nommés GT, les groupes de travail sont des réunions où se discute les évolutions à l'université. Dernièrement plusieurs ont eu lieu : GT mise en place du portail unique, GT fonctionnement du conseil ST, GT calendrier  universitaire 2022-2023, GT contrôle de connaissances en licence (groupe disciplinaire). Ces réunions ne sont pas périodiques mais arrivent lors de la mise en place de chantiers ou pour faire un point "RETEX" (retour d'expérience).

J'ai participé au groupe de travail retour d'expérience sur la mise en place du portail unique et le ressenti des usagers (= étudiants) après cette première année. Mon rôle était d'apporter un regard de l'intérieur du portail, avec un discussion sur les "groupes classes" qui changent pour chaque matière. Nous avons également formulé des améliorations qui pourraient être apportées pour faciliter l'intégration des nouveaux entrants étudiants, la gestion administrative (inscription pédagogique/administrative) et des salles. Des pistes d'améliorations pour 2023-2024 (outils numériques principalement) ont menées à la mise en place d'autres réunions avec la DSI de l'université.

Bien sûr, le mail reste le principal outil de travail du conseiller, et plus encore avec la pandémie. Il permet l'organisation de réunions, la discussion, le partage de documents, ... C'est une tâche de fond durant tout le mandat : pour rester au fait des dernières affaires, questions et soucis de la communauté, se tenir au courant des réformes en cours, des chantiers, des directions données par l'université et le MESRI (Ministère de l'Enseignement Supérieur et de la Recherche) et, jusqu'à il y a encore quelques semaines, des directives COVID-19 à l'université.

Depuis décembre et plus généralement depuis mon entrée à l'université, mon utilisation des mails a beaucoup augmentée. Le plus flagrant étant ma capacité de rédaction qui s'est améliorée, surtout dans le cadre des fonctions d'élus. Mon vocabulaire a quand à lui évolué grâce cette immersion prolongée dans le monde de l'administration universitaire publique. Pour donner un exemple, le terme marché publique m'est aujourd'hui beaucoup plus familier. 

\textit{Le plus gros travail qui se fait c'est encore l'apprentissage des abréviations spécifiques ;)}

Cet ensembles d'activités m'a aussi permis de développer des compétences oratoires. Ainsi qu'une vision plus "gestionnaire" qu'avant. 

\begin{center}
\textbf{Répartition approximative des heures d'implications en tant qu'élu étudiant}
\vspace*{10pt}

\begin{tabular}{|c|c|c|}
\hline 
Activités & Périodicité & Temps total\\ 
\hline 
Conseil de collège ST & Tous le mois & 5*4h = 20h \\ 
\hline 
Réunions préparatoires & Tous les mois & 5*1h (direction collège) + 4*1h (candidats) = 9h\\ 
\hline 
Groupes de travail & Non périodique & 1*3h + 1h = 4h\\ 
\hline 
Mail & Plusieurs fois/semaine & 20 semaines*1h/semaine = 20h\\ 
\hline 
\textbf{Total} &  & \textbf{53h} (approximativement) \\ 
\hline 
\end{tabular} 
\end{center}
\textit{NB : * signifie multiplication ; h signifie heure(s)}

\end{document}